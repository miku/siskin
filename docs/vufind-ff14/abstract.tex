\documentclass[german]{article}

\usepackage{graphicx}
\usepackage{hyperref}

\setlength\parindent{0pt}

\usepackage[sc]{mathpazo}
\linespread{1.05}         % Palatino needs more leading (space between lines)
\usepackage[T1]{fontenc}

\usepackage[utf8]{inputenc}

\begin{document}
\thispagestyle{empty}
\section*{Integration von Normdaten und anderen Datenquellen in ein Discoverysystem -- von Rohdaten zu Services}

\paragraph{Referenten}  \href{mailto:anke.hofmann@hmt-leipzig.de}{Anke Hofmann}, HMT Leipzig und \href{mailto:martin.czygan@uni-leipzig.de}{Martin Czygan}, UB Leipzig\\
\vspace{0.2cm}

Es gibt eine Reihe von offenen Datenquellen,
die Verknüpfungspunkte zu bibliothekarischen Daten aufweisen und daher
auch für Discoverysysteme interessant sein können. Neben der
zentralen GND (Gemeinsame Normdatei) finden sich weitere Datenpools,
wie DBPedia\footnote{\url{http://dbpedia.org}}, Wikidata\footnote{\url{http://www.wikidata.org}},
VIAF (Virtual International Authority File)\footnote{\url{http://viaf.org/}}
oder Freebase\footnote{\url{https://www.freebase.com/}} -- die maschinenlesbare
Informationen über Medien, Personen, Geografika, Institutionen oder Ereignisse
enthalten. Wir wollen in einem Vortrag vorstellen, wie man von den
Rohdaten über verschiedene Zwischenschritte zu Services kommt, die in
Discoverysysteme eingebunden werden können. Vorgestellt werden die Datenquellen,
Ansätze zur Prozessierung der heterogenenen Formate und Inhalte,
HTTP-Schnittstellen und deren Nutzung für die Einbindung in das Frontend.
Der Votrag soll zum Anlass genommen werden, die Public-Beta Version
eines Knowledge-Discovery Services zu starten.

\end{document}
