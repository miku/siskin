% Copyright 2014 GESELLSCHAFT FÜR INFORMATIK E.V. (GI) GI-EV.DE
\documentclass[english]{lni}

\usepackage{graphicx}
\usepackage{hyperref}

\author{
Martin Czygan \\
Leipzig University Library \\
Beethovenstra\ss{}e 6 \\
04107 Leipzig \\
martin.czygan@uni-leipzig.de
}
\title{Design and Implementation of a Library Metadata Management Framework and its Application
in Fuzzy Data Deduplication and Data Reconciliation with Authority Data}
\begin{document}
\maketitle

\begin{abstract}
We describe the application of a generic workflow management
system to the problem of metadata processing in the library domain. The
requirements for such a framework and acting real-world forces
are examined. The design of the framework is layed out and illustrated by means of two example workflows: fuzzy data deduplication and data reconciliation
with authority data. Fuzzy data deduplication is the process of finding
similar items in record collections that can not be matched by identifiers. Data reconciliation with authority
data takes a data source and enhances the metadata of its records -- mainly authors and subjects -- with available
normed data. Finally, the advantages and tradeoffs of the presented approach are discussed.

\end{abstract}

\section{Introduction}

The system described in this paper is a result of a few
product iterations on the problem of library metadata management. The problem
emerged during project \emph{finc}\footnote{More information about the project can
be found on its homepage at \url{http://finc.info} and in~\cite{GBV0007757838}.} -- a project for the development of search engine based discovery interfaces for university libraries in Saxony -- where a couple of heterogeneous data sources needed to be processed and merged in certain, sometimes convoluted, ways. The main contribution of this paper is to report on core requirements (section~\ref{requirements}) of such a system, and how most of these requirements can be met by adapting a generic workflow management framework (sections~\ref{forces} and~\ref{design}). Subsequently, two example workflows from the library domain are described in short (section~\ref{examples}), followed by a discussion (section~\ref{discussion}), where key tradeoffs are identified.

\section{Related Work}

There are a number of open source software products for workflow
management, dependency graph handling and data processing.
A prominent example of a tool that utilizes a dependency graph is make\footnote{\url{http://www.gnu.org/software/make/}}, a build tool for software projects\footnote{Other open source projects with similar scope are -- among others -- rake, ant, maven, gradle, scons, waf.}.
More data-oriented projects include
Oozie\footnote{From \url{https://oozie.apache.org/}: ``Oozie is a workflow scheduler system to manage Apache Hadoop jobs.
Oozie Workflow jobs are Directed Acyclical Graphs (DAGs) of actions.''},
Askaban\footnote{From \url{https://azkaban.github.io/}: ``Azkaban is a batch workflow job scheduler created at LinkedIn to run Hadoop jobs. Azkaban resolves the ordering through job dependencies and provides an easy to use web user interface to maintain and track your workflows.''}, joblib\footnote{From \url{https://pythonhosted.org/joblib/}: ``Joblib is a set of tools to provide lightweight pipelining in Python.''}, Taverna\footnote{From \url{http://www.taverna.org.uk/}: ``Taverna is an open source and domain-independent Workflow Management System -- a suite of tools used to design and execute scientific workflows and aid in silico experimentation.''} and luigi\footnote{From \url{https://github.com/spotify/luigi}: ``Luigi is a Python package that helps you build complex pipelines of batch jobs. It handles dependency resolution, workflow management, visualization, handling failures, command line integration, and much more.''}.
Features that made luigi a compelling choice\footnote{The system design is decribed in more detail in section~\ref{design}.} as the base
for our system are a) a small code base\footnote{luigi 1.0.16 contains 5125 LOC, which included the task abstractions, a scheduler, support for hdfs, hadoop, hive, scalding, postgres, Amazon S3, date, configuration and compression handling and task histories.}, b) efficient command line integration and c) a single, lightweight notation to express tasks and their relationships.

\section{Requirements}
\label{requirements}

An easily formulated requirement of a metadata management application
for libraries is the ability to work with different data sources and to
combine, filter and transform the data, as it traverses the system. This requirement
implies some kind of formalism of a unit, on which transformations can be applied.
Typical notions for these units would be a file on the filesystem or on the web, a single
record or a stream of records. This requirement is broad and leaves a lot of design decisions untouched. To add additional design constraints, we enumerate a few non-functional requirements.

\paragraph{Consistency}
Transformations on data should happen in a transactional manner
or should in other ways be guaranteed to be atomic\footnote{\cite{haerder1983principles} describes atomicity as one of the four characteristics of a transaction: ``Atomicity. It must be of the all-or-nothing type [...], and the user must, whatever happens, know which state he or she is in.''}, that is a transformation
should either be applied to a data element or not. Atomicity should be
enforced at different data granularity levels, from a single field in a
single record, that is rewritten, to sets of potentielly
tens of millions of records, that need a transformation to be
applied on all of its elements.

\paragraph{Composability}
It is desireable to design the data elements or the
processing units in ways that support reuse.
There are several approaches to this: A certain data package can be used to build various
workflows atop. In that way, workflows do not need to recompute
certain states or shapes of the data over and over again. Another way of reuse
would be the application of a single task or processing unit on
several data streams. For example, different data sources could be fetched
in form of compressed files from an FTP server, so the task of fetching and
extracting these files would present itself as a template task, because
for several different data sources the processing would be similar up to a fixed number of parameters.

\paragraph{Efficiency} Many data sources are updated in intervals and
recomputations of data derived from that data source in turn should
happen as frequently as the data source itself is updated. In general,
a workflow, that encompasses a number of data sources should be executed
as soon as any one of the constituent data sources changes in a significant
way. Conceptually, one can view the combination of a number of data sources as a data source
itself with the update times consisting of the update times
of all subsumed data sources along with a total order.

\paragraph{Ease of deployment, agile development and change management} The application
should be easily relocatable\footnote{This is certainly interesting in a cloud setup,
where machines may be spun up and down on a regular basis.}
and should bootstrap its
data environment as autonomously as possible. This proves to be useful
for development since a simple setup enables developers to
mirror a production environment on a development machine and therefore
reduce the gap between both setups.

Development of new workflows
and changes to existing workflows should not affect established setups, that
might be part of a production system. It should be possible to distribute
tasks sensibly among members of a team.

\paragraph{Testability, transparency and introspection}
Writing code that deals with millions of records and complicated workflows
can be a challenge, as would be writing unit tests for those workflows. It would
be desireable to be able to unit test components as well as to write integration
tests for complete workflows with the help of mock services.

Most workflow
management systems will break down a multi-stage workflow into
several subtasks. One should be able to introspect the shape of the
data along the way at the different steps to support transparency and debuggability.

\section{Forces}
\label{forces}

There are a number of forces around the problem of library metadata management,
of which a few should be introduced in this section.

\paragraph{Data formats}

The format of the data to be processed varies widely. While there are standardized data
formats for libraries such as MARC\footnote{\url{http://www.loc.gov/marc/}},
MAB\footnote{\url{http://www.dnb.de/DE/Standardisierung/Formate/MAB/mab_node.html}} or RDA\footnote{\url{http://www.rda-jsc.org/rda.html}},
there is still enough room for variation. Other data sources, which
might be of interest to the library community, have not yet been
standardized or are not necessarily curated with the goal of standardization.
Typical formats are XML (which in turn can come in many flavors to
express similar things), JSON, HTML, N-Triples, CSV or plain text. All
these formats can furthermore be compressed or span multiple files.

\paragraph{Data access} As we described for the data formats the accessibility of the data varies widely, too.
Typical examples are FTP servers, metadata harvesting protocols such as
OAI-PMH~\cite{oai2008}, Web-APIs, E-Mail attachments, database or index
connections.

\paragraph{Update intervals and update signalling}
\label{updates}

Typical for the use case, data sources are updated in intervals.
The update intervals can range from a
few seconds over days to many weeks, months or years. In other words
they can range from real-time to a singular update.
In these remarks of we do not consider real-time, but only cases, where
the update granularity does not underrun 24 hours\footnote{The workflow
framework ships with a parameters class, that allows update intervals of 60 minutes. Below
that, the inherently batch-oriented nature of the framework might not prove supportive.}.

An interesting observation concerns the way, the availability of an update
is signalled. This can happen in many ways as well, with two broad categories being explicit or implicit.
An example for explicit signalling would be a file name containing a date
or some other indicator of succession, such as a serial number. Implicit
signalling means that the data source will not convey the
availability of updates, unless it is preprocessed to
reveal this information. An example for implicit signalling would be
OAI-harvesting, where one would need to perform an actual request and a
comparison with existing records to find out, whether updates are available\footnote{A \emph{from} parameter in the request is required, which in turn can only be derived from some previous harvest.}.

Another aspect during update processing is the way in which an update
to a data source is expressed. There are at least two ways in which this can be
achieved in practice. A simple way to express an update to a data source is to provide a complete dump.

Using deltas is also common practice. Given an initial dump or a delta against an empty state,
subsequent changes to the data source are expressed as changes, which
need to be applied to the previous state of the data source. The upside
of this approach is the limitation of the update to the affected
records, the downside -- which sometimes results in significant
overhead in the amount of processing -- is the need to keep all the metadata
packages available if one wants to recompute some state of the data
source after a number of deltas. The way in which the deltas are formulated can differ
between data sources. One data source might provide an
update or deletion list as an extra file, while other data sources
ship updated or deleted records with a flag set at a predefined record field. The
variety of these notions does not assist in the definition
of a consistent abstraction layer, when one tries to generalize the
event of an update over a bigger number of data sources\footnote{This fact suggests,
that the notion of an update to a data source alone might not be
the abstraction level sweet spot.}.

\paragraph{Processing speed}
It is desirable to utilize modern multicore
computers by implementing systems, which take advantage of parallel processing
whenever feasible. It should be noted that while certain parts of a workflow
might not be suitable for parallelization, others might and the processing
framework would ideally be able to identify these parts by itself
and schedule the tasks accordingly.

\section{Design}
\label{design}

In this section, the overall system design is outlined along with the
description of how the implementation addresses the requirements and forces.

\paragraph{Building blocks}

The system described in this report is based upon an open source software library called
luigi\footnote{\url{https://github.com/spotify/luigi}},
developed by Spotify\footnote{``Luigi was built at Spotify, mainly by Erik Bernhardsson and Elias Freider, but many other people have contributed.``, \url{https://github.com/spotify/luigi}} --
a music streaming service. This library provides
basic abstractions for workflow management and execution\footnote{The framework comes
with more than just the basic abstractions. Most notable there are wrappers around HDFS~\cite{borthakur2008hdfs} and
Hadoop~\cite{white2009hadoop} to support cluster computing as well.} and embodies many of
the desired properties listed in section~\ref{requirements}. The contribution
of this report is the description of how this generic framework for
implementing workflows can be adapted for the library domain.

The basic building block of the framework is the notion of a task, which is
both simple and powerful. A task has a couple of essential properties: It
has zero, one or more requirements, which are tasks themselves; furthermore
a task has a body of code, which corresponds to the business logic that needs to
be executed; finally every task has one or more targets\footnote{While it is possible to create tasks with multiple targets, it is recommended, that each task has exactly one artefact as output.}.
Aside these three core notions each task has zero or more parameters.
Every workflow is decomposed into a number of tasks. Tasks are implemented
and wired together in the source code itself. Figure~\ref{impression}
gives an impression of an emerging depedency graph. Workflow execution is realized through command line integration, which supports
interactive use as well as automation with standard tools such as cron\footnote{
\url{http://unixhelp.ed.ac.uk/CGI/man-cgi?cron}}.

A central scheduler, implemented as a deamon, handles task scheduling and visualisation. While the notion of a target supports many different implementations\footnote{The community implemented targets for Amazon Redshift, MySQL, Postgres, Elasticsearch.},
the majority of the artefacts are files.

\begin{figure}[ht!]
\centering
\includegraphics[scale=0.43]{depgraph.pdf}
\caption{An impression of a dependency graph (with simplified task names and without task parameters).
Different sources will differ in their internal complexity,
depending on various factors, such as update intervals and signalling. Workflows can
be built upon tasks from many sources.}
\label{impression}
\end{figure}

\paragraph{Consistency}

The framework itself does not enforce consistency, but makes it easy
to implement it. Following the rule, that each task will create an
artefact only if the task has been run successfully, it is possible
to carry over the consistency from tasks to workflows. A workflow (which is
just a set of tasks) is successfully completed, only if every
task executed successfully. When dealing with files, the atomicity
can be enforced in the task by using the POSIX rename interface~\cite{posix2004}\footnote{``This rename() function is equivalent for regular files to that defined by the ISO C standard. Its inclusion here expands that definition to include actions on directories and specifies behavior when the new parameter names a file that already exists. That specification requires that the action of the function be atomic.``}.

On the other side, it is possible to repeatedly execute workflows.
Since each task has a defined target (e.g. a file), it will not execute
the task again, if it has been run successfully in the past (e.g. if the output file exists) with the same
set of parameters.

Idempotence and atomicity in the building blocks
aid in reducing the number of states the system
can be in, which benefits the resilience of the overall system and also
helps to reason about what happened and what failed.

\paragraph{Composability}
\label{composability}

The notion of a task underlines the provision of small units. While there is
no limitation in what a task can do and how many lines of code it can contain,
it is convenient to keep the amount of work and responsibility for a
single task limited. In an evaluation of about $300$ tasks, that
implement several different workflows in our system, the average number
of lines of code (LOC) inside the body of a task was $13.7$. Figure~\ref{locfreq} shows
the distribution of the lines of code over the tasks.
The number of tasks that constitute a workflow ranges between
a few tasks up to several hundreds of tasks.

The nature of the framework supports composability also by letting the
developer implement generic tasks, that can be reused by other tasks by
adding these generic tasks to the workflow or derive special tasks
from them. Examples for generic tasks, that have been implemented in our system are
tasks for mirroring FTP files and directories, harvesting OAI-PMH interfaces
or indexing documents.

\begin{figure}[htb]
\begin{center}
\includegraphics[width=10cm]{locdist}
\caption{\label{locfreq}Distribution of lines of code (LOC) over 317 tasks. The average LOC per task (business logic only) is $13.7$.}
\end{center}
\end{figure}

\paragraph{Efficiency}

Two important types of efficiency are efficiency
concerning tasks and efficiency in the context of parallelism. Efficiency
with regard to tasks means that tasks and workflows are only executed,
if there has been an update to the data source in question. As described
in section~\ref{updates}, the update intervals can vary widely between data sources.
It can be sufficient to parameterize a task by date and to have it
execute in regular intervals (e.g. in weekly, monthly or quarterly intervals).
Other tasks might have some latest state, which can only be evaluated
by looking at the data source itself. For these cases, a special parameter
class has been implemented, that is able to map any given date to the closest
date in the past, on which an update occured for a data source. In this way a task can
respond with an correct output artefact for arbitrary dates. The business logic of the task
is only executed if absolutely necessary, therefore contributing to the efficiency of the overall
setup.

The second type of efficiency is handled by the framework, which can use any
number of workers. A worker corresponds to a separate OS process. Since the tasks
in the system form a dependency graph, the framework is able to determine,
which tasks are parallelizable and will spawn separate processes to execute them, given the user allows
more the one worker to run.

\paragraph{Ease of deployment, agile development and change management}
\label{ease}

Most of the system's artefacts are files, which are organized under a single
directory. This makes it relatively easy to backup or replicate the metadata --
both original and derived -- to other machines by copying a single directory.
The software consists of a single python package that can be installed with
standard tools. One strength of the setup is the ability to almost
completely bootstrap a metadata infrastructure, given a valid configuration
has been provided. This, again, is possible because the tasks form a dependency graph.
This feature supports agile development as well. A developer does not
need to know, how a certain state or artefact of the metadata (e.g. a list
of ISBNs from a few different data sources or a list of records from a data source,
that have been updated in the last three weeks) is provided, as long as there
is a task in the system that provides that specific artefact.

During development it has been our observation, that certain tasks are referenced more often
than other. One could call these tasks \emph{stable points}, since they underpin
more workflows than other tasks. On the other hand, there are tasks, that merely
serve as an intermediate step in some longer computation, making them
less important in the overall setup. These kind of metrics\footnote{Each
task records its performance and reports it to the user. The benchmarks
are implemented with python decorators, making it easy to extend this system
into a metrics database that could monitors performance characteristics
over time, should the need arise.}
can also inform development. We saw that investing time in the refinement of the stable points in the
architecture yielded performance benefits for many workflows, whereas it was
often possible to contract less important intermediate steps into a single larger
task for the sake of performance -- without sacrificing too much conceptual
clarity by agglutinating many steps within a single task.

\paragraph{Testability, transparency and introspection}

As described in section~\ref{composability}, the average number of LOC for a
task is relatively low. Furthermore, a task has a defined set of inputs (often
just a single file) and outputs (per recommendation just a single file).
This helps writing unit tests, since a task resembles a function with
a number of arguments and most of the time no or only a few references to
any kind of state outside itself. Since the tasks themselves can be tested
with moderate efforts, complicated workflows, which span hundreds of tasks,
become manageable as well.

The mostly file-based setup of the system facilitates another aspect: transparency.
Most tasks produce a single output and this output has a location\footnote{We extended the built-in file system abstractions of the framework to be able to almost complety hide the underlying directory structure from the developer or user.} on the file system.
In our system, we wrote small wrapper scripts, that help to inspect tasks with a single command, abstracting
from the real filename, referring to the task only by its name and parameters\footnote{This feature is especially useful, when a new data source is added to the system
and needs to be explored with the help of a few initial tasks.}. Another
addition concerns documentation. By combining the documentation facilities\footnote{\url{http://legacy.python.org/dev/peps/pep-0257/}}
and the dynamic nature of the python programming language with the implicit
dependency information encoded in the task, it is possible to autogenerate
comprehensive command line documentation, similar to a UNIX man page\footnote{\url{http://manpages.bsd.lv/history.html}}.

\paragraph{Deployment setup}

The software is a single python package, splitted into two modules: One contains
the data sources, another the higher level workflows. The data source
modules take care of the data from its origin (e.g. FTP servers, web sites, databases),
to some stable point, e.g. in our case this is mostly a single JSON file that
represents the current state of the data source or a corresponding Elasticsearch index.
The workflow modules work
with those data sources, modeling a variety of things, such as data deduplication,
reporting or data enrichment and reconciliation. The python package itself
has a couple of external dependencies for database access, index access,
ISBN handling, tabular data manipulation among other things. Common tasks
and utilities have been factored out into a separate package.

\section{Example workflows}
\label{examples}

Two workflows, that have been implemented atop various data sources are
briefly described in this section. The important observation in both examples
is the fact, that non-trivial tasks, that originate in the library data domain,
can be made manageable, by builing upon a powerful data structure like
a dependency graph.

\subsection{Fuzzy data deduplication}

As a first example we describe briefly the process of fuzzy data deduplication.
We selected three data sources with about five million records as a first target.
There are three workflows involved to get each data source
into an Elasticsearch index. Two data sources work with regularly updated
files on FTP-Servers, one data source uses a more involved scheme involving
dumps, deltas and additional lists for deletions. Of the two data sources that use
FTP, one conveys updates in form of complete dumps, whereas the other uses
a set of deltas. As described in section~\ref{ease}, a stable point in our setup
is an Elasticsearch index with a document structure, that is relatively homogenous
across data sources. These indices are kept up-to-date as the data source commands it.

For the fuzzy deduplication a dual approach of cosine and ngram similarity is
used. First a list of ids is generated, for which similar items are collected
with the help of the a heuristic query\footnote{This query uses Lucene's \emph{MoreLikeThis} facility which uses a \emph{bag-of-words} model to find similar documents.}, based on record titles.
The titles and authors of the candidates are then broken into trigrams
and compared. Records with a similarity score above a threshold are then
reported as fuzzy duplicates.  Additional fields are evaluated in a postprocessing step. Records, for example,
that do not share a similar edition statement are removed from the list of
duplicates\footnote{The definition of a duplicate will vary in different situations.}.

\subsection{Data reconciliation with authority data}

Another requirement concerns data enrichment and data reconciliation.
Since our setup uses over 30 data sources, the quality of the metadata
is not necessary alike. Especially there are data sources, that can benefit from
the knowledge already encoded in form of library metadata in other
data sources. One example is the use of high quality authority data produced and
published by the German National Library in form of the Integrated Authority File (GND)~\cite{behrens2011gemeinsame}.
This authority file contains among other things naming variants for entities,
for example authors or composers. On the other hand there are data sources
in our metadata landscape, that carry a single naming variant of a person,
thus making it impossible to find the record by using a valid name that
is not part of the record. The workflow here takes the entries from the GND
and successively compares the naming variants to names and entities found in the
less rich metadata source, generating suggestions for metadata enrichment.
In a quality control postprocessing step, the suggestions are checked for plausibility, by
evaluating additional information about the subject\footnote{We implemented this workflow for the University of Music and Theatre Leipzig.
We could utilize information about the profession of a person, encoded in the GND taxonomy, to
weigh suggestions. For example there are many people named Bach, of which not all are composers. In a second step,
we used biographical data, also encoded in the GND, to further clean the list of suggestions.}.

\section{Discussion}
\label{discussion}

There are a couple of tradeoffs involved in a setup, as described in this
report. The first and probably most significant is the space overhead
of a file based approach. Similar to a map-reduce~\cite{dean2008mapreduce} style processing, where
each intermediate step is persisted to disk, before another phase in the
workflow picks it up, the disk space overhead is considerable. Depending
on the concrete workflows, the increase in storage requirements in our
setup can be up to tenfold -- compared to the size of the raw original data
source. It is possible to view the data generated by the tasks of the workflow
as a cache, computing parts of the process once and reducing the number
of computations required for subsequent steps. One example for this incremental
computing\footnote{\emph{Reactive Imperative Programming with Dataflow Constraints}~\cite{demetrescu2011reactive}
describes incremental computing as follows: ``When the input is subject to small changes,
a program may incrementally fix only the portion of the output affected by the update, without having to recompute the
entire solution from scratch. ``} approach is the handling of data sources, that ship updates with deltas.
In one case, each delta package needs a few processing steps (e.g. extraction,
conversion, filter). In another step all deltas and their derived artefacts
are considered to create a snapshot of the data source. The amount of work
that needs to be done when a new delta arrives is minimal, once the derived artefacts
of the previous deltas are in place -- making it possible to quickly
create snapshots of that data source.

This extensive caching facilitates the solution of other problems in a~straightforward manner. One example would be the computation of change
reports for data sources. Since many stages of the metadata has been or
can be recomputed, the comparison of the state of the metadata between
two arbitrary time instances becomes surprisingly simple. One just calls
the same task with two different date parameter values and extracts the difference\footnote{If,
for example, the two tasks emit single column data, one can load the two artefacts
into two sets and perform a set difference.}.

A disadvantage of the presented approach is the discipline required to
implement the tasks. Each tasks must be written in an manner, so that it
supports atomicity and idempotency. Also not every task is suitable for
this functional paradigm, although from empirical evidence we are able
to conclude, that many different real world tasks are. Persisting intermediate steps can lead to inefficiencies, since
performance could benefit from keeping data in memory for a certain
amount of time or for a certain chain of tasks\footnote{This line of
critique seems to apply to the map-reduce style framework Hadoop as well~\cite{spark2014}: ``Spark is able to execute batch-processing jobs between 10 to 100 times faster than the MapReduce engine according to Cloudera, primarily by reducing the number of writes and reads to disc.``
Another }.

\section{Summary}
\label{summary}

In summary, the advantages of a system like the one described in this
report are the following: A workflow composed of a number of smaller tasks
facilitates testable, reusable components. If it is easy to reason
about a single task, complex workflows spanning multiple data sources
can stay manageable. Supplying tasks with parameters for time can help with the isolation
of update behaviours and can shield the developer from the idiosyncrasies of a data source
by providing stable points in the architecture, which enable a multitude
of workflows. Finally, trading extensive space requirements for a gain in
simplicity and composability seems to be a worthwhile tradeoff in the library metadata domain.

\bibliography{tr-dilmmf}

\end{document}
